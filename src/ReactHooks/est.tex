Estado: permite que componentes criem e gerenciem seus proprio dados, um exemplo são os diferentes estados,aparência, de sites dureante o proecesso de carragamento,
determina oque deve ocorer em diferentes momentos da aplicação
Podemos ter varios estados por componentes, quando qualquer um deles dfor alterado o corpo sera recarragado

Hooks: permitem automatizar o processo de checagem de estado, identifica quando um valor é alterado dentro do componente e recarega o mesmo em tela
Reatividade:

Ciclo de vida do componente: todo componente tem um ciclo de vida, o 1ª começa quando ele é construido em tela;
2ª momento é quando ele é atualizado; 3ª momento é quando ele é removido da tela,deixando de existir dentro da nossa interface;

UserEfect: existe para ajudar a lidar com a alteração do ciclo de vida, exemplo:ter um código que queremos executar só no 1ªmomento do ciclo,
ou caso eu queira rodar um código toda vez que o estado do programa muda, ou caso eu queira executar um programa quando o componente sai da minha tela