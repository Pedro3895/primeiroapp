UserEfect
import logo from './logo.svg';
import './App.css';
import React, { useState, useEffect } from 'react';//use = padrão de hooks importados do react e criados por nos
import Button from './Components/button/button';
import Header from './Components/header/header';


function App() {
 
  //const hookState = useState(true)//hookState é uma função nativa do react, onde devemos passar o valor inicial do meu estado
  const [carregando, setCarregando] = useState(true)//1ªvalor atual:;2ª:função que altera o valor set+valor atual
  //estado do contador; o hook pode ser qualquer coisa, neste caso um number
  //contador é a constante em sí e SetContador é uma função, por isso podemos passa-los para dentro de outro componente como propriedades
  const [contador, setContador] = useState(0)

  useEffect(()=>{
    console.log('Carragou pela primeira vez')

  }, [])//é um metodo/função, dentor tera a função que queremos executar seguido do array de dependencias
//o array de dependencias contem os estados de dependencias, que informam o userEfect da atualização do estado, caso deixe vazio só sera executado na primeira renderização
  return (
   <div>
    {carregando ? 
    <span>Carregando...</span> 
    :
    <div>
      <button onClick = {()=> setContador(contador+1)}>Adicionar</button>
      <span>{contador}</span>
    </div>
    }
    <button onClick = {()=>setCarregando(!carregando)}>{carregando ? 'Carregar Site' : 'Voltar para carregamento'}</button>
    <Button name='Alterar Valor' onClick={setContador} />
    </div>
  );
}
//spam mostra o número exibido
//onClik para adicionar um evento de adicionar +1 ao contador
//clicando no button o onclick leva o setContador para dentro do Button.js, oque torna o onClick no setContador
//clicar no button altera o valor por +1, ao clicar no Button o contador mostra 20 
export default App;
//<Button name='Paulo'/>passamos o valor do atributo='Paulo'
//mesmos componentes transformados pelas propriedades
//só o active ele entende como true

2ª momento do ciclo:
import logo from './logo.svg';
import './App.css';
import React, { useState, useEffect } from 'react';//use = padrão de hooks importados do react e criados por nos
import Button from './Components/button/button';
import Header from './Components/header/header';


function App() {
 
  //const hookState = useState(true)//hookState é uma função nativa do react, onde devemos passar o valor inicial do meu estado
  const [carregando, setCarregando] = useState(true)//1ªvalor atual:;2ª:função que altera o valor set+valor atual
  //estado do contador; o hook pode ser qualquer coisa, neste caso um number
  //contador é a constante em sí e SetContador é uma função, por isso podemos passa-los para dentro de outro componente como propriedades
  const [contador, setContador] = useState(0)

  useEffect(()=>{
    console.log('Carragou pela primeira vez')//será executado quando for renderizado da 1ªvez e quando o contador tiverum novo valor

  }, [contador])//contador é dependencia da userEfect

  return (
   <div>
    {carregando ? 
    <span>Carregando...</span> 
    :
    <div>
      <button onClick = {()=> setContador(contador+1)}>Adicionar</button>
      <span>{contador}</span>
    </div>
    }
    <button onClick = {()=>setCarregando(!carregando)}>{carregando ? 'Carregar Site' : 'Voltar para carregamento'}</button>
    <Button name='Alterar Valor' onClick={setContador} />
    </div>
  );
}
//spam mostra o número exibido
//onClik para adicionar um evento de adicionar +1 ao contador
//clicando no button o onclick leva o setContador para dentro do Button.js, oque torna o onClick no setContador
//clicar no button altera o valor por +1, ao clicar no Button o contador mostra 20 
export default App;
//<Button name='Paulo'/>passamos o valor do atributo='Paulo'
//mesmos componentes transformados pelas propriedades
//só o active ele entende como true

multiplos userEfect
import logo from './logo.svg';
import './App.css';
import React, { useState, useEffect } from 'react';//use = padrão de hooks importados do react e criados por nos
import Button from './Components/button/button';
import Header from './Components/header/header';


function App() {
 
  //const hookState = useState(true)//hookState é uma função nativa do react, onde devemos passar o valor inicial do meu estado
  const [carregando, setCarregando] = useState(true)//1ªvalor atual:;2ª:função que altera o valor set+valor atual
  //estado do contador; o hook pode ser qualquer coisa, neste caso um number
  //contador é a constante em sí e SetContador é uma função, por isso podemos passa-los para dentro de outro componente como propriedades
  const [contador, setContador] = useState(0)

  useEffect(()=>{
    console.log('Contador')//será executado quando for renderizado da 1ªvez e quando o contador tiverum novo valor

  }, [contador])//
//posso ter infinitos estados como dependencia do meu userEfect, mais é mais comum criar vários userEfect's
  
  useEffect(()=>{ 
    console.log('Carregando')
  }, [carregando])

  return (
   <div>
    {carregando ? 
    <span>Carregando...</span> 
    :
    <div>
      <button onClick = {()=> setContador(contador+1)}>Adicionar</button>
      <span>{contador}</span>
    </div>
    }
    <button onClick = {()=>setCarregando(!carregando)}>{carregando ? 'Carregar Site' : 'Voltar para carregamento'}</button>
    <Button name='Alterar Valor' onClick={setContador} />
    </div>
  );
}
//spam mostra o número exibido
//onClik para adicionar um evento de adicionar +1 ao contador
//clicando no button o onclick leva o setContador para dentro do Button.js, oque torna o onClick no setContador
//clicar no button altera o valor por +1, ao clicar no Button o contador mostra 20 
export default App;
//<Button name='Paulo'/>passamos o valor do atributo='Paulo'
//mesmos componentes transformados pelas propriedades
//só o active ele entende como true

3ª momento do cicloimport logo from './logo.svg';
import './App.css';
import React, { useState, useEffect } from 'react';//use = padrão de hooks importados do react e criados por nos
import Button from './Components/button/button';
import Header from './Components/header/header';


function App() {
 
  //const hookState = useState(true)//hookState é uma função nativa do react, onde devemos passar o valor inicial do meu estado
  const [carregando, setCarregando] = useState(true)//1ªvalor atual:;2ª:função que altera o valor set+valor atual
  //estado do contador; o hook pode ser qualquer coisa, neste caso um number
  //contador é a constante em sí e SetContador é uma função, por isso podemos passa-los para dentro de outro componente como propriedades
  const [contador, setContador] = useState(0)

  useEffect(()=>{
    console.log('Contador')//será executado quando for renderizado da 1ªvez e quando o contador tiverum novo valor

  }, [contador])//
//para quando o componente é retirado da tela, devemos passar o callback(função) dentro da nossa função de callback já estabelecida
//uma função dentro de outra função, iremos retornar o callback quando o elemento deve ser deconstruido  
  useEffect(()=>{ 
    console.log('Carregando')

    return () =>{//executado quando o trecho de código é removido da tela
      

    }

  }, [carregando])

  return (
   <div>
    {carregando ? 
    <span>Carregando...</span> 
    :
    <div>
      <button onClick = {()=> setContador(contador+1)}>Adicionar</button>
      <span>{contador}</span>
    </div>
    }
    <button onClick = {()=>setCarregando(!carregando)}>{carregando ? 'Carregar Site' : 'Voltar para carregamento'}</button>
    <Button name='Alterar Valor' onClick={setContador} />
    </div>
  );
}
//spam mostra o número exibido
//onClik para adicionar um evento de adicionar +1 ao contador
//clicando no button o onclick leva o setContador para dentro do Button.js, oque torna o onClick no setContador
//clicar no button altera o valor por +1, ao clicar no Button o contador mostra 20 
export default App;
//<Button name='Paulo'/>passamos o valor do atributo='Paulo'
//mesmos componentes transformados pelas propriedades
//só o active ele entende como true